\chapter*{謝辞}
\addcontentsline{toc}{chapter}{謝辞}
本論文は明治大学自律型ロボット研究クラスターの下で実施されました. ここに厚くお礼を申し上げます. \par
指導教員である黒田先生からは研究に対する向き合い方を始めとして様々なことを教えていただきました. 黒田先生からはロボティクスに関する知識だけでなく, 様々なことを学ばせていただきました. 特に自分の研究を人に伝えるための姿勢は私の口頭での学会発表や就職活動において非常に役に立ち, 研究室生活を通じて最も成長した部分であると実感しています. 来年度から研究室を取り巻く環境も良い方向へ大きく変わるとともに, なにかと忙しくなるかと思われますがお体に気を付けてお過ごしください. 黒田研の益々の発展を期待しております. \par
松田先生には私が学部4年生の時に黒田先生とともに研究の面倒を見ていただきました. 毎回のミーティングや論文の添削など, 様々な場面で頂いたアドバイスは当時研究について右も左も分からなかった私にとって非常にありがたかったとともに, 私が研究に関する知識や技術を涵養していく上での非常に強力な支えとなってくれました. 松田先生が所属する機関での研究, マルチロボットグループでの研究とご多忙の中のご指導本当にありがとうございます. 松田先生は昨年度明治大学の理工学部に着任され, 来年度から学生の研究室配属が始まります. 松田研の今後の益々の発展を期待しております. \par
株式会社小川優機製作所様は私が修士1年の際に行った共同研究でロボットの開発に携わらせていただきました. 共同研究を通して涵養された仕事に対する責任感は, 4月から自動車会社の研究所に就職する私にとって大きな財産となりました. 共同研究の中では私の至らぬ点も多くありましたが, その中でも小川優機の皆様が私にかけて頂いた支えの言葉はとても嬉しかったです. 株式会社小川優機製作のご発展をお祈り申し上げます. \par
研究室の先輩方からは研究に関わる知識や技術を教えていただいた他に, 研究に対して抱いている高い意識を学ばせていただきました. 黒田研は理工学部の中でも優秀な学生が集まる研究室であると言われています. これは黒田研の歴代の先輩方が持っていた研究に対する高いモチベーションと豊かな学習意欲が結びついた結果であると私は考えております. 大学院生活を通じて指針となる極めて優秀な先輩方と巡り会えたとともに, その先輩方から得られた多くのことに対してここでお礼を申し上げます. \par
同期の5人に対しては研究室内だけに留まらない様々な場面で時間をともに過ごすとても心強い仲間でありました. 研究に関する議論に留まらず, プライベートな相談など様々な面でサポートしていただきました. そのおかげで約3年間の長く密で, そして楽しい時を私は過ごすことができました. ここに感謝を申し上げます. \par
研究室の後輩たちに対しては, 実験やTA, つくばチャレンジなど様々な場面で助けていただきました. 先ほども触れたように, 黒田研の学生の良いところは研究に対する高いモチベーションと豊富な学習意欲にあり, 今の黒田研の後輩たちは皆漏れなくその良い部分を持っていると私は思っています. 来年度から黒田研を取り巻く環境は大きく変わりますが, その中でも先ほど述べた「黒田研らしさ」は非常に大きな強みとしてあり続け, 時に助けになると考えています. これからもその「黒田研らしさ」をこれから入ってくる学生たちに伝えていただけたら幸いです.\par
私が研究室に配属された直後から新型コロナウイルスの世界的大流行が始まり, そこから長い間対面の学会に参加することも叶いませんでしたが, 修士2年の最後に2つの対面の学会に参加できた経験は私にとってとても良い経験となれました. 特に最後のアトランタでの国際学会に参加できた経験は一生の良い思い出となりました. 学会参加にあたって様々な面でサポートしていただいた中川さんにここでお礼申し上げます.\par
最後に, 24年間にわたって生活をサポートしていただき, 大学生活を通して一切不自由なく学ぶことのできる環境を整えていただいた家族への感謝をもってこの修士論文を締めたいと思います.


\begin{flushright}
    明治大学理工学研究科 \\
    機械工学専攻 \\
    ロボット工学研究室修士 2 年 \\
    2023年 2月 河合響 \\
\end{flushright}
