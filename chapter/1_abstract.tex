\chapter*{概要}
\addcontentsline{toc}{chapter}{概要}
自律移動ロボットの中には姿勢を自在に傾けることで高い機動力を発揮することのできるものが存在する. このようなロボットにとって自らの姿勢を正確に把握することは必須条件である. 姿勢推定の手法はドローンや航空機の分野で盛んに研究されている. これらの手法は主に加速度センサなどの内界センサからの値を積分して姿勢を推定するものであり, 自律移動ロボットではタイヤが地面と接することによって発生する振動によって誤差蓄積の問題が発生するため使用することができない. この問題を解決する上で重要な役割を果たすのがカメラなどの外界センサである. 外界センサを用いた姿勢推定を行うことで内界センサによって発生する誤差蓄積を解消することができるとともに, より高度な姿勢推定を可能にすることができる. \par
本研究ではカメラ画像とDNNを利用する姿勢推定手法について提案する. カメラ画像とそれに対応する姿勢角度を学習したDNNによって姿勢推定を行うことで内界センサが抱える誤差蓄積の問題を解決し, 時系列で採取された画像を入力データ用いることでロボットが持つ運動の特性を学習してより高度な姿勢推定を可能とする. 本研究ではシミュレータにおける事前学習と実環境で採取されたデータを用いたFine Tuningによって実環境で動作する姿勢推定精度を示す.
