\chapter{結論}
本論文では時系列画像を入力とするDNNを用いた実環境で動作する姿勢推定を提案した. 提案手法は実験において, ワンショットで推定していた既存手法と比較して有効な推定精度を発揮できなかったものの, 時系列画像の採取周期が変動しても推定精度に大きな変化が起きないといった実環境で使用する上では強いメリットとなる特性が示された. 過去にシミュレータで採取された大量の訓練用画像を用いて学習したネットワークを用いた研究ではシミュレータ内で高い推定精度が示されており, 学習に用いるデータの量の改善と学習手法のさらなる改良を行うことでより実用的な推定精度が得られることが強く予想される. \par
本論文で提案された姿勢推定手法は実運用に耐える姿勢推定性能を有していることが強く予想されるため, 今後のさらなる改良が施されることでより実用的な推定手法が研究されることが期待される.